\documentclass[
  jou,
  longtable,
  nolmodern,
  notxfonts,
  notimes,
  colorlinks=true,linkcolor=blue,citecolor=blue,urlcolor=blue]{apa7}

\usepackage{amsmath}
\usepackage{amssymb}



\usepackage[bidi=default]{babel}
\babelprovide[main,import]{english}


% get rid of language-specific shorthands (see #6817):
\let\LanguageShortHands\languageshorthands
\def\languageshorthands#1{}

\RequirePackage{longtable}
% \setlength\LTleft{0pt}
\RequirePackage{threeparttablex}

% % % 
% % 


\makeatletter
\renewcommand{\paragraph}{\@startsection{paragraph}{4}{\parindent}%
	{0\baselineskip \@plus 0.2ex \@minus 0.2ex}%
	{-.5em}%
	{\normalfont\normalsize\bfseries\typesectitle}}

\renewcommand{\subparagraph}[1]{\@startsection{subparagraph}{5}{0.5em}%
	{0\baselineskip \@plus 0.2ex \@minus 0.2ex}%
	{-\z@\relax}%
	{\normalfont\normalsize\bfseries\itshape\hspace{\parindent}{#1}\textit{\addperi}}{\relax}}
\makeatother




\usepackage{longtable, booktabs, multirow, multicol, colortbl, hhline, caption, array, float}
\setcounter{topnumber}{2}
\setcounter{bottomnumber}{2}
\setcounter{totalnumber}{4}
\renewcommand{\topfraction}{0.85}
\renewcommand{\bottomfraction}{0.85}
\renewcommand{\textfraction}{0.15}
\renewcommand{\floatpagefraction}{0.7}

\usepackage{tcolorbox}
\tcbuselibrary{listings,theorems, breakable, skins}
\usepackage{fontawesome5}

\definecolor{quarto-callout-color}{HTML}{909090}
\definecolor{quarto-callout-note-color}{HTML}{0758E5}
\definecolor{quarto-callout-important-color}{HTML}{CC1914}
\definecolor{quarto-callout-warning-color}{HTML}{EB9113}
\definecolor{quarto-callout-tip-color}{HTML}{00A047}
\definecolor{quarto-callout-caution-color}{HTML}{FC5300}
\definecolor{quarto-callout-color-frame}{HTML}{ACACAC}
\definecolor{quarto-callout-note-color-frame}{HTML}{4582EC}
\definecolor{quarto-callout-important-color-frame}{HTML}{D9534F}
\definecolor{quarto-callout-warning-color-frame}{HTML}{F0AD4E}
\definecolor{quarto-callout-tip-color-frame}{HTML}{02B875}
\definecolor{quarto-callout-caution-color-frame}{HTML}{FD7E14}


% 

\newlength\Oldarrayrulewidth
\newlength\Oldtabcolsep


\usepackage{hyperref}




\providecommand{\tightlist}{%
  \setlength{\itemsep}{0pt}\setlength{\parskip}{0pt}}
\usepackage{longtable,booktabs,array}
\usepackage{calc} % for calculating minipage widths
% Correct order of tables after \paragraph or \subparagraph
\usepackage{etoolbox}
\makeatletter
\patchcmd\longtable{\par}{\if@noskipsec\mbox{}\fi\par}{}{}
\makeatother
% Allow footnotes in longtable head/foot
\IfFileExists{footnotehyper.sty}{\usepackage{footnotehyper}}{\usepackage{footnote}}
\makesavenoteenv{longtable}

\usepackage{graphicx}
\makeatletter
\def\maxwidth{\ifdim\Gin@nat@width>\linewidth\linewidth\else\Gin@nat@width\fi}
\def\maxheight{\ifdim\Gin@nat@height>\textheight\textheight\else\Gin@nat@height\fi}
\makeatother
% Scale images if necessary, so that they will not overflow the page
% margins by default, and it is still possible to overwrite the defaults
% using explicit options in \includegraphics[width, height, ...]{}
\setkeys{Gin}{width=\maxwidth,height=\maxheight,keepaspectratio}
% Set default figure placement to htbp
\makeatletter
\def\fps@figure{htbp}
\makeatother


% definitions for citeproc citations
\NewDocumentCommand\citeproctext{}{}
\NewDocumentCommand\citeproc{mm}{%
  \begingroup\def\citeproctext{#2}\cite{#1}\endgroup}
\makeatletter
 % allow citations to break across lines
 \let\@cite@ofmt\@firstofone
 % avoid brackets around text for \cite:
 \def\@biblabel#1{}
 \def\@cite#1#2{{#1\if@tempswa , #2\fi}}
\makeatother
\newlength{\cslhangindent}
\setlength{\cslhangindent}{1.5em}
\newlength{\csllabelwidth}
\setlength{\csllabelwidth}{3em}
\newenvironment{CSLReferences}[2] % #1 hanging-indent, #2 entry-spacing
 {\begin{list}{}{%
  \setlength{\itemindent}{0pt}
  \setlength{\leftmargin}{0pt}
  \setlength{\parsep}{0pt}
  % turn on hanging indent if param 1 is 1
  \ifodd #1
   \setlength{\leftmargin}{\cslhangindent}
   \setlength{\itemindent}{-1\cslhangindent}
  \fi
  % set entry spacing
  \setlength{\itemsep}{#2\baselineskip}}}
 {\end{list}}
\usepackage{calc}
\newcommand{\CSLBlock}[1]{\hfill\break\parbox[t]{\linewidth}{\strut\ignorespaces#1\strut}}
\newcommand{\CSLLeftMargin}[1]{\parbox[t]{\csllabelwidth}{\strut#1\strut}}
\newcommand{\CSLRightInline}[1]{\parbox[t]{\linewidth - \csllabelwidth}{\strut#1\strut}}
\newcommand{\CSLIndent}[1]{\hspace{\cslhangindent}#1}




\usepackage{newtx}

\defaultfontfeatures{Scale=MatchLowercase}
\defaultfontfeatures[\rmfamily]{Ligatures=TeX,Scale=1}





\title{\vskip 1.5cm
\bfseries Explicit Weight Bias Concerns in the Fitness Industry: A Quantitative Analysis}
\shorttitle{Template for the apaquarto Extension}


\usepackage{etoolbox}


\journal{\bfseries \Large Journal of Kinesiology and Wellness}
\volume{\vskip 1mm \normalsize A Publication of the Western Society for
Kinesiology and Wellness \newline Volume 13, Number 1, Pages 1--9,
2024\newline ISSN\# 2323-4505}





\authorsnames{Kellie Walters,Alison Ede}




\affiliation{
{Department of Kinesiology, California State University, Long Beach}}






\leftheader{Walters and Ede}



\abstract{Limited and conflicting research is available regarding weight
biases in the fitness industry, yet implications of such biases are
pervasive. Individuals in larger bodies often experience stigma and
prejudices due to their weight, and anti-fat attitudes have been
normalized in the fitness industry and associated educational pipelines.
Current literature on weight biases in the fitness industry lacks
context and fails to examine these biases from an intersectional lens.
Therefore, this study explored how social identities (e.g., age, gender,
race, etc.) influence weight biases in fitness professionals. Fitness
professionals completed an electronic survey that included demographic
questions and measures of weight bias (Anti-Fat Attitudes Test; AFAT)
and body dissatisfaction (Contour Drawing Rating Scale). Women in the
healthy (2.02 ± .51) and overweight (1.97 ± .49) BMI categories had
significantly greater total AFAT scores (p = .003 and p = .023,
respectively) compared to women in the obese BMI category (1.63 ± .48).
For participants who had completed some college, those who were
classified in the healthy BMI category had significantly greater total
AFAT scores (2.05 ± .50) compared to those in the overweight BMI
category (1.72 ± .46). For participants who completed a master's degree,
those in the healthy BMI category (2.08 ± .56) and overweight BMI
category (2.05 ± .43) had significantly greater total AFAT scores
compared to those in the obese BMI category (1.48 ± .46). There was a
direct effect of gender, body dissatisfaction, race and BMI on AFAT
subscales. There was also a significant direct effect of body
dissatisfaction on AFAT subscales. Across all variables, AFAT scores
were highest for the physical subscale (2.69 ± .91) and lowest for the
social subscale (1.43 ± .45). Fitness professionals exhibit explicit
weight biases, and future research should examine the implications of
such biases.}
% 
\keywords{Anti-fat bias, fitness professionals, social
identities, explicit bias, weight stigma}

\authornote{\par{\addORCIDlink{Kellie
Walters}{0000-0002-0892-8028}}\par{\addORCIDlink{Alison
Ede}{0000-0001-8144-0013}}
\par{ }
\par{       }
\par{Correspondence concerning this article should be addressed
to Kellie Walters, Department of Kinesiology, California State
University, Long Beach, 1250 Bellflower Blvd, HHS2-221, Long
Beach, CA 90840, Email: kellie.walters@csulb.edu}
}


\usepackage{pbalance} 
\usepackage{float}
\makeatletter
\let\oldtpt\ThreePartTable
\let\endoldtpt\endThreePartTable
\def\ThreePartTable{\@ifnextchar[\ThreePartTable@i \ThreePartTable@ii}
\def\ThreePartTable@i[#1]{\begin{figure}[!htbp]
\onecolumn
\begin{minipage}{0.5\textwidth}
\oldtpt[#1]
}
\def\ThreePartTable@ii{\begin{figure}[!htbp]
\onecolumn
\begin{minipage}{0.5\textwidth}
\oldtpt
}
\def\endThreePartTable{
\endoldtpt
\end{minipage}
\twocolumn
\end{figure}}
\makeatother


\makeatletter
\let\endoldlt\endlongtable		
\def\endlongtable{
\hline
\endoldlt}
\makeatother

\newenvironment{twocolumntable}% environment name
{% begin code
\begin{table*}[!htbp]%
\onecolumn%
}%
{%
\twocolumn%
\end{table*}%
}% end code

\urlstyle{same}





\begin{document}

\maketitle

\setcounter{secnumdepth}{-\maxdimen} % remove section numbering

\setlength\LTleft{0pt}


Weight bias (i.e., anti-fat bias) is unreasonable judgments about
someone based on weight Washington (2011) (Washington, 2011). It is
pervasive in the health industry, including those who work as physicians
(Schwartz et al., 2003), physical educators (Fontana et al., 2017),
fitness professionals (Dimmock et al., 2009; Fontana et al., 2018;
Robertson \& Vohora, 2008), and exercise science students (Chambliss et
al., 2004; Fontana et al., 2013; Langdon et al., 2016; Rukavina et al.,
2010; Wijayatunga et al., 2019). In the fitness industry, potential
implications of these biases include negative perceptions of larger
bodied individuals' abilities, motivation, and potential job
qualifications (Sartore \& Cunningham, 2007). Weight stigma is defined
as discriminatory acts towards individuals in larger bodies due to their
size (Washington, 2011). Consequences of experiencing weight stigma
include a) poor physical health, such as an increased likelihood of
maintained obesity or weight gain (Sutin \& Terracciano, 2013), and b)
increased psychological distress, including greater rates of body
dissatisfaction and symptoms of eating disorders (Vartanian \& Novak,
2011). Paradoxically, individuals who experience weight stigma are more
likely to avoid exercise as a result of internalized anti-fat attitudes
(Vartanian \& Novak, 2011) and experience an increased allostatic load
(cumulative response to ongoing stress) ((Guidi et al., 2021), which has
a greater impact on their health than being in a larger body does
(Gordon, 2020; Milburn et al., 2019).

A systematic review on weight bias among exercise and nutrition
professionals included 31 studies; however, only three focused
specifically on fitness professionals (e.g., personal trainers or group
fitness instructors) compared to ``exercise professional trainees''
(e.g., exercise science students). Robertson and Vohora (2008) were the
first to report strong anti-fat implicit and explicit biases in fitness
professionals (n = 57, ``gym instructors'' and ``aerobics
instructors''), with the bias being greater in those who had never been
overweight and believed obesity was controllable. In a study surveying
fitness center employees (management and administrative staff n = 15,
personal trainers n = 16, fitness instructors n =19, and exercise/sport
physiologists n = 20), Dimmock et al.~(2009) reported a moderately
strong implicit bias, but no explicit bias, towards individuals in
larger body sizes. More recently, Fontana et al.~(2018) found that
personal trainers (n = 52) report strong implicit biases against
individuals who are obese.

Recently, Zaroubi et al.~(2021) published a review article on the
predictors of weight bias in fitness professionals and exercise science
students (Zaroubi et al., 2021). Most of the studies in this review
sampled undergraduate students in the exercise science field, with only
four of the 18 sample fitness professionals. Of those four studies, only
three included weight bias as a dependent variable (Dimmock et al.,
2009; Fontana et al., 2018; Robertson \& Vohora, 2008). A thematic
analysis was conducted, and six themes emerged. First, exercise science
students and fitness professionals strongly believe that weight is
controllable and associate individuals with larger bodies with negative
attributes such as laziness. Second, the relationship between gender and
weight bias is still unknown as data is conflicting. Third, being
enrolled in an exercise science or similar educational program is likely
a predictor of weight bias. Fourth, personal and psychosocial factors
(e.g., the tendency to internalize an athletic body as the ideal body
shape) likely influence weight bias. Fifth, knowledge of the
uncontrollable aspects of obesity (e.g., genetics) is likely to lower
weight bias. Lastly, there is conflicting evidence regarding the
influence of one's personal history with someone in a larger body.
Chambliss et al.~(2004) report that a lack of family history of having a
larger body leads to higher explicit weight bias in fitness
professionals and regular exercisers Field (Chambliss et al.(Chambliss
et al., 2004). In contrast, DeBarr and Pettit reported no statistical
differences in weight bias held by health educators classified as
overweight compared to normal weight. Little research has examined
explicit weight biases of fitness professionals, and no research has
focused on whether their social identities and/or role in the industry
(e.g., group fitness instructor versus personal trainer) influence their
weight bias. This research is particularly important due to the
influential nature of this field. Clients often look to fitness
professionals for advice and education on changing their health
behaviors. If fitness professionals hold strong weight biases, they may
contribute to a harmful cycle whereby their clients become less likely
to participate and/or adhere to their health behavior changes. Fitness
professionals need to have more knowledge of weight biases. Thus, the
study aimed to examine the influence of age, gender, body
dissatisfaction, race, role in industry, BMI, income, and education on
weight bias in fitness professionals.

\section{Methods}\label{methods}

\subsection{Participants}\label{participants}

The original data set included participants (n = 366) who identified as
fitness professionals in various settings. Participants reported their
role in the industry with the option to choose from certified personal
trainer (n = 30), group fitness instructor (n = 107), facility club
manager/director/owner (n = 2), physical/occupational therapist (n = 2),
health/wellness coach (n = 2), strength and conditioning coach (n = 4),
other with the option to enter their role (n = 15), and multiple (those
who hold more than one role in the industry; n = 189). Due to low sample
sizes within some of the roles (facility club manager,
physical/occupational therapy, health/wellness coach, strength and
conditioning coach, and others), only data from individuals who marked
that they were personal trainers, group fitness instructors, or those
who held multiple roles were included in the analysis (n = 326). The
participants included a diverse sample, with 40.5\% identifying as
non-white (11\% Black, 6.1\% Asian, 8.6\% Hispanic, 3.1\% other, and
9.5\% multi-race) and 59.5\% identifying as White. Participants
identified as female (n = 262) and male (n = 55), and their age was
relatively equally distributed across all age groups ranging from 18-55+
(21.2\% 18-24 years old, 26.1\% 25-34 years old, 22.4\% 35-44 years old,
17.2\% 45-54 years old, and 12.9\% 55 years old and older). Participants
were well educated (66\% having a minimum bachelor's degree), and 43.7\%
reported an annual household income of \$100,000 or more.

Recruitment occurred via word of mouth, email, and social media.
Participants were asked to complete an electronic survey about weight
biases in the fitness industry. IRB approval and written participant
consent were received before data collection. After four months of data
collection, the authors recognized that the majority of respondents up
until that point were white (84\%) and subsequently amended the IRB
application to include an incentive (\$20 gift card) for individuals of
color to participate in the study. After adjusting the recruitment
language to include information about the incentive, an additional 109
fitness professionals who identified as persons of color completed the
survey.

\subsection{Instruments}\label{instruments}

In addition to demographic data (participants' age, weight, height, BMI,
gender, race, education, income, and role in the industry), the
following instruments were used in this study.

\subsubsection{Anti-fat Attitudes Test
(AFAT)}\label{anti-fat-attitudes-test-afat}

The modified 34-item AFAT scale (Lewis et al., 1997) measured explicit
bias attitudes towards individuals in larger bodies (i.e., weight bias
or anti-fat bias). This psychometrically sound scale (Dimmock et al.,
2009; Lewis et al., 1997; Wijayatunga et al., 2019) consisted of a
modified 5-point Likert scale with 1 being strongly disagree and 5 being
strongly agree. To avoid social response bias, participants were
reminded multiple times that their responses were anonymous. Positively
worded statements were reverse coded, so higher scores represented
greater anti-fat bias. The questionnaire includes three subscales: (1)
social/character disparagement (e.g., ``I prefer not to associate with
fat people''), (2) physical/romantic unattractiveness (e.g., ``Fat
people are physically unattractive''), and (3) weight control/blame
(e.g., ``There is no excuse for being fat''), as well as a total
composite score (Lewis et al., 1997). Individual questions were averaged
for each subscale and the total AFAT composite score. Cronbach's alpha
was .71, .78, and .71 for the social, attraction, and blame subscales,
respectively, indicating adequate internal consistency.

\subsubsection{Contour Drawing Rating
Scale}\label{contour-drawing-rating-scale}

The psychometrically sound Contour Drawing Rating Scale assessed
participants' body dissatisfaction (Gardner \& Brown, 2010). As
introduced by Thompson and Gray (1995), the Contour Drawing Rating Scale
utilizes the drawings of masculine and feminine human figures in the
front view. Nine drawings illustrate each gender, with illustrations
representing progressively larger body shapes on a scale of 1 to 9.
First, the participants chose which body type they most identified with
(e.g., ``Which bodies do you mostly identify with?''), with group A
being body shapes traditionally assigned to women and group B being body
shapes traditionally assigned to men (Figure 1). As noted earlier, this
data assessed participants' gender identity. The participants answered
two more questions including: (1) ``On a scale from 1-9, rate what your
CURRENT body size based on the images above'', and (2) ``On a scale from
1-9, rate what you would IDEALLY want to look like based on the images
above.'' Participants' body dissatisfaction was calculated by
subtracting the number associated with their ideal image from the number
associated with their current image.

Positive scores indicated a desired ideal body smaller than their
current perceived body size, and negative scores indicated an ideal body
larger than their current perceived body size. Scores ranged from -2 to
4 and were categorized into four groups: (1) moderate dissatisfaction,
desire to be larger (scores of -2 and -1), (2) no dissatisfaction
(scores of 0, meaning their current body size was their ideal body
size), (3) moderate dissatisfaction, desire to be smaller (scores of 1
and 2), and (4) high dissatisfaction, desire to be smaller (scores of 3
and 4). While body dissatisfaction as a construct represents a desire to
be a different shape, creating groups distinguishing positive and
negative scores allows for a more nuanced understanding of body
dissatisfaction. A desire to be smaller should represent a greater
internalization of weight bias than a desire to be larger, as a desire
to have a smaller body size is consistent with the societal ideal that
thinner is more valued.

\subsection{Statistical Analysis}\label{statistical-analysis}

Two-way ANOVAs were conducted to examine the effects of every possible
2-way interaction of the eight IVs (age, gender, body dissatisfaction,
race, BMI, role, education, and income) on AFAT total (Tables 1-6). When
no interaction effects were found, one-way ANOVAs and MANOVAs were
conducted to assess the direct effect of the IVs on AFAT total and AFAT
subscales, respectively. Partial eta squared (partial η²) was used to
measure the effect size of variables, with 0.1, 0.06, and 0.14
indicating a small, medium, and large effect size, respectively (Fritz
et al., 2012). Outliers were assessed by inspection of a boxplot,
normality was assessed using Shapiro-Wilk's normality test for each cell
of the design, and Levene's test assessed homogeneity of variances. All
data is presented as mean ± standard deviation. SPSS Version 27 was used
to analyze the data, and significance was noted by a p-value \textless{}
0.05.

\section{Results}\label{results}

There was a statistically significant interaction between gender and BMI
on total AFAT scores,
\(F(2,272) = 3.139, p = .045, \text{partial } \eta^2 = .023\).
Therefore, an analysis of simple main effects for gender and BMI was
performed with statistical significance receiving a Bonferroni
adjustment. Women in the healthy \((2.02 \pm .51)\) and overweight
\((1.97 \pm .49)\) BMI categories had significantly greater total AFAT
scores \((p = .003 \text{ and } p = .023, \text{respectively})\)
compared to women in the obese BMI category \((1.63 \pm .48)\). There
was also a statically significant interaction between education and BMI
on total AFAT scores,
\(F(9,266) = 2.201, p = .022, \text{partial } \eta^2 = .069\). An
analysis of simple main effects for education and BMI was performed with
statistical significance receiving a Bonferroni adjustment. For
participants who had completed some college, those who were classified
in the healthy BMI category had significantly greater total AFAT scores
\((2.05 \pm .50)\) compared to those in the overweight BMI category
\((1.72 \pm .46), p = .045\). For participants who completed a master's
degree, those in the healthy BMI category \((2.08 \pm .56)\) and
overweight BMI category \((2.05 \pm .43)\) had significantly greater
total AFAT scores \((p = .003 \text{ and } .016, \text{respectively})\)
compared to those in the obese BMI category \((1.48 \pm .46)\). No other
interaction effects were found. Therefore, one-way ANOVAs and MANOVAs
were conducted to assess the direct effect of the IVs on AFAT total and
AFAT subscales, respectively. The mean total AFAT scores for each IV are
listed in Table 7.

\subsection{Age}\label{age}

A one-way ANOVA was conducted to determine if total anti-fat bias
differed by age group. Homogeneity of variances was violated, as
assessed by Levene's Test of Homogeneity of Variance \((p = .001)\).
Therefore, Welch's F and Games-Howell were used to assess significance.
There were no statistically significant differences in total AFAT scores
between the different age groups, Welch's F
\((4, 128.682) = 0.632, p = .640\). A one-way MANOVA was run to
determine if anti-fat bias subscale scores differed by age group. Across
all age groups, AFAT scores were highest for the physical subscale
\((2.69 \pm 0.92)\) and lowest for the social subscale
\((1.43 \pm 0.45)\), with the blame subscale scores between
\((2.17 \pm 0.74)\). No statistically significant differences existed
between age groups for all three AFAT subscales,
\(F(12,852) = 1.415, p =.153; \text{Pillai’s Trace} = .059; \text{partial } \eta^2 = .020\).

\begin{ThreePartTable}

\begin{longtable}[]{@{}
  >{\raggedright\arraybackslash}p{(\columnwidth - 8\tabcolsep) * \real{0.5000}}
  >{\raggedright\arraybackslash}p{(\columnwidth - 8\tabcolsep) * \real{0.1300}}
  >{\raggedright\arraybackslash}p{(\columnwidth - 8\tabcolsep) * \real{0.1300}}
  >{\raggedright\arraybackslash}p{(\columnwidth - 8\tabcolsep) * \real{0.1300}}
  >{\raggedright\arraybackslash}p{(\columnwidth - 8\tabcolsep) * \real{0.1300}}@{}}
\caption{Summary of Two-way ANOVAs by Age}\label{tbl-tb1}\tabularnewline
\toprule\noalign{}
\begin{minipage}[b]{\linewidth}\raggedright
Measure
\end{minipage} & \begin{minipage}[b]{\linewidth}\raggedright
\emph{F}
\end{minipage} & \begin{minipage}[b]{\linewidth}\raggedright
\emph{df}
\end{minipage} & \begin{minipage}[b]{\linewidth}\raggedright
\emph{p}
\end{minipage} & \begin{minipage}[b]{\linewidth}\raggedright
\(\eta_{p}^{2}\)
\end{minipage} \\
\midrule\noalign{}
\endfirsthead
\toprule\noalign{}
\begin{minipage}[b]{\linewidth}\raggedright
Measure
\end{minipage} & \begin{minipage}[b]{\linewidth}\raggedright
\emph{F}
\end{minipage} & \begin{minipage}[b]{\linewidth}\raggedright
\emph{df}
\end{minipage} & \begin{minipage}[b]{\linewidth}\raggedright
\emph{p}
\end{minipage} & \begin{minipage}[b]{\linewidth}\raggedright
\(\eta_{p}^{2}\)
\end{minipage} \\
\midrule\noalign{}
\endhead
\bottomrule\noalign{}
\endlastfoot
Gender X race & 0.26 & 4 & 0.905 & 0.00 \\
Gender X age & 1.27 & 4 & 0.280 & 0.02 \\
Gender X education & 0.08 & 4 & 0.988 & 0.00 \\
Gender X income & 0.60 & 7 & 0.757 & 0.02 \\
Gender X BMI & 3.14 & 2 & 0.045* & 0.02 \\
Gender X body dissatisfaction & 0.18 & 3 & 0.908 & 0.00 \\
Gender X industry role & 0.53 & 2 & 0.592 & 0.00 \\
\end{longtable}

{\noindent \emph{Note.} Note below table}

\end{ThreePartTable}

\subsection{Gender}\label{gender}

A one-way ANOVA was conducted to determine if total anti-fat bias
differed by gender. There was a statistically significant difference in
total AFAT scores between genders, \(F(1, 280) = 8.320, p = .004\).
Participants who identified as men reported significantly greater total
AFAT scores \((2.19 \pm 0.41)\) than those who identified as women
\((1.96 \pm 0.51)\). A one-way MANOVA was run to determine if anti-fat
bias subscales differed by gender. For both genders, AFAT scores were
highest for the physical subscale \((2.71 \pm 0.92)\) and lowest for the
social subscale \((1.43 \pm 0.46)\), with the blame subscale scores
between \((2.17 \pm 0.74)\). The differences between genders on the AFAT
physical subscale
\(F(1,280) = 6.940, p < .05; \text{partial } \eta^2 = .024\) and AFAT
blame subscale
\(F(1,280) = 6.909, p < .05; \text{partial } \eta^2 = .024\) were
statistically significant. Participants who identified as men reported
significantly greater AFAT physical scores \((3.02 \pm .91; p<.05)\) and
AFAT blame scores \((2.42 \pm .62; p<.05)\) than participants who
identified as women
\((2.64 \pm .91 \text{ and } 2.12 \pm .75, \text{respectively})\).

\begin{ThreePartTable}

\begin{longtable}[]{@{}llrc@{}}
\caption{My Table}\label{tbl-mytable2}\tabularnewline
\toprule\noalign{}
Default & Left & Right & Center \\
\midrule\noalign{}
\endfirsthead
\toprule\noalign{}
Default & Left & Right & Center \\
\midrule\noalign{}
\endhead
\bottomrule\noalign{}
\endlastfoot
12 & 12 & 12 & 12 \\
123 & 123 & 123 & 123 \\
1 & 1 & 1 & 1 \\
\end{longtable}

{\noindent \emph{Note.} This is a note below the markdown table.}

\end{ThreePartTable}

\subsection{Body Dissatisfaction}\label{body-dissatisfaction}

A one-way ANOVA was conducted to determine if total anti-fat bias
differed by body dissatisfaction. There were statistically significant
differences in total AFAT scores between different levels of body
dissatisfaction, \(F(3, 276) = 4.147, p < .05\). Those participants in
the moderate dissatisfaction, desire to be smaller group
\((1.94 \pm .50)\) had significantly lower total AFAT scores compared to
those in the no dissatisfaction group \((2.15 \pm .49), p = .017\).

A one-way MANOVA was run to determine if anti-fat bias subscales
differed by body dissatisfaction. AFAT scores were highest for the
physical subscale \((2.71 \pm 0.92)\) and lowest for the social subscale
\((1.42 \pm 0.44)\), with the blame subscale scores between
\((2.17 \pm 0.74)\). The effect of body dissatisfaction on AFAT
subscales was statistically significant,
\(F(9,828) = 2.010, p < .05; \text{Pillai’s Trace} = .064; \text{partial } \eta^2 = .021\).
Tukey post-hoc tests showed that participants in the moderate
dissatisfaction, desire to be smaller group \((1.36 \pm .42)\) had
significantly lower AFAT social scores than participants in the no
dissatisfaction group \((1.57 \pm .49), p = .004\).

\subsection{Race}\label{race}

A one-way ANOVA was conducted to determine if total anti-fat bias
differed by race. Homogeneity of variances was violated, as assessed by
Levene's Test of Homogeneity of Variance \((p = .001)\). Therefore,
Welch's F and Games-Howell were used to assess significance. Total AFAT
scores were statistically significantly different by race, Welch's F
\((5,47.989) = 7.564, p < .001\). White participants reported
significantly lower total AFAT scores \((1.86 \pm .51)\) than Black
\((2.14 \pm .40, p = .008)\), Asian \((2.25 \pm .34, p = .001)\), and
Hispanic \((2.25 \pm .38, p <.001)\) participants. A one-way MANOVA was
run to determine if anti-fat bias subscales differed by race. AFAT
scores were highest for the physical subscale \((2.68 \pm 0.92)\) and
lowest for the social subscale \((1.42 \pm 0.44)\), with the blame
subscale scores between \((2.17 \pm 0.74)\). The effect of race on AFAT
subscales was statistically significant,
\(F(15,834) = 7.813, p < .001; \text{Pillai’s Trace} = .370; \text{partial } \eta^2 = .123\).
Tukey post-hoc tests demonstrated that participants who identified as
White reported significantly lower AFAT physical scores
\((2.26 \pm .72)\) than participants who identified as Black
\((3.46 \pm .69, p<.001)\), Asian \((3.12 \pm .71, p<.001)\), Hispanic
\((3.30 \pm .87, p<.001)\), other \((3.46 \pm .81, p<.001)\), and
multi-race \((3.04 \pm 1.02, p<.001)\).

\subsection{BMI}\label{bmi}

A one-way ANOVA was conducted to determine if total anti-fat bias
differed by BMI. There were statistically significant differences in
total AFAT scores between different levels of BMI,
\(F(2, 282) = 3.278, p < .05\). Those participants in the healthy BMI
category \((2.06 \pm .51)\) had significantly greater total AFAT scores
compared to those in the obese BMI category
\((1.79 \pm .53), p = .034\). A one-way MANOVA was run to determine if
anti-fat bias subscales differed by BMI. AFAT scores were highest for
the physical subscale \((2.71 \pm 0.91)\) and lowest for the social
subscale \((1.44 \pm 0.45)\), with the blame subscale scores between
\((2.19 \pm 0.74)\). There were significant differences in the AFAT
subscale scores between BMI categories,
\(F(6, 560) = 2.126, p = .049; \text{Pillai’s Trace} = .044; \text{partial } \eta^2 = .022\).
Tukey post-hoc tests demonstrated that participants in the healthy BMI
category reported significantly greater AFAT blame scores
\((2.27 \pm .70)\) than participants in the obese BMI category
\((1.76 \pm .77, p = .003)\).

\subsection{Role in Industry}\label{role-in-industry}

A one-way ANOVA was conducted to determine if total anti-fat bias
differed by role in the industry. There were no statistically
significant differences in total AFAT scores between roles,
\(F(2, 287) = 0.487, p = .615\). A one-way MANOVA was run to determine
if anti-fat bias subscales differed by role in the industry. AFAT scores
were highest for the physical subscale \((2.69 \pm 0.92)\) and lowest
for the social subscale \((1.43 \pm 0.50)\), with the blame subscale
scores between \((2.17 \pm 0.74)\). There were no significant
differences in any of the AFAT subscale scores between the different
industry roles,
\(F(6, 572) = 1.60, p = .142; \text{Pillai’s Trace} = .033; \text{partial } \eta^2 = .017\).

\subsection{Education}\label{education}

A one-way ANOVA was conducted to determine if total anti-fat bias
differed by education level. There were no statistically significant
differences in total AFAT scores between education levels,
\(F(6, 282) = 1.528, p = .169\). A one-way MANOVA was run to determine
if anti-fat bias subscales differed by education level. AFAT scores were
highest for the physical subscale \((2.69 \pm 0.92)\) and lowest for the
social subscale \((1.43 \pm 0.45)\), with the blame subscale scores
between \((2.17 \pm 0.74)\). There were no statistically significant
differences in the AFAT subscale scores between different levels of
education,
\(F(18,846) = 0.083, p = .158; \text{Pillai’s Trace} = .083; \text{partial } \eta^2 = .028\).

\subsection{Income}\label{income}

A one-way ANOVA was conducted to determine if total anti-fat bias
differed by income level. Homogeneity of variances was violated, as
assessed by Levene's Test of Homogeneity of Variance \((p = .028)\).
Therefore, Welch's F and Games-Howell were used to assess significance.
There were no statistically significant differences in total AFAT scores
between the different income groups, Welch's F
\((7, 108.923) = 1.472, p = .185\). A one-way MANOVA was run to
determine if anti-fat bias subscales differed by income level. AFAT
scores were highest for the physical subscale \((2.70 \pm 0.92)\) and
lowest for the social subscale \((1.43 \pm 0.45)\), with the blame
subscale scores between \((2.17 \pm 0.73)\). There were no statistically
significant differences in the AFAT subscale scores between different
levels of income,
\(F(21,819) = 1.39, p = .141; \text{Pillai’s Trace} = .100; \text{partial } \eta^2 = .033\).

\section{Discussion}\label{discussion}

A key finding from this study is that a fitness professional's BMI
interacts with their gender and education level in relation to total
weight bias. Women in the healthy and overweight BMI category had
significantly greater total weight bias compared to those in the obese
BMI category. This supports previous findings that higher BMI is
associated with lower anti-fat bias (Elran-Barak \& Bar-Anan, 2018;
Marini et al., 2013). These findings suggest that being in a smaller
body increases women's likelihood of exhibiting greater anti-fat biases.
Results from this study demonstrate that those with more education
reported greater weight biases in smaller (i.e., healthy BMI category)
and larger bodies (i.e., overweight BMI category) than those with less
education who only reported greater weight biases in smaller bodies.
Previous research has found that as physical education students advance
in their degrees, they will likely demonstrate greater weight bias
(O'Brien et al., 2007; Wijayatunga et al., 2019). Therefore, a
relationship between education level and severity of weight biases may
exist whereby, despite being in a larger body, students with more
education may be more likely to hold negative beliefs about individuals
in larger bodies due to internalizing the weight biases inherent in
health education (Zaroubi et al., 2021).

Results from this study and others demonstrate that men reported
significantly greater anti-fat biases when controlling for BMI than
women (Chambliss et al., 2004; Langdon et al., 2016). From a societal
perspective, it is more acceptable for men to be in larger bodies than
women (Heise et al., 2019). However, when BMI was included in this
study's analysis, women with smaller bodies were the ones who held
greater anti-fat biases. The drive for thinness that is perpetuated in
the fitness industry may contribute to women internalizing messaging
about what it means to be in a larger body, which may contribute to
having a greater anti-fat bias.

Both females and males in the healthy BMI category reported
significantly greater total weight bias compared to those in the obese
BMI category. These findings result from cultural norms where larger
bodies are often viewed as less desirable than smaller bodies. This is
seen in the fitness industry's lack of images of people with larger
bodies and the current culture of dismissing people as less able, less
self-disciplined, and lacking willpower (Foster et al., 2003; Hebl \&
Xu, 2001). Interestingly, fitness professionals in this study with
moderate body dissatisfaction and a desire to be smaller reported
significantly lower total and social weight bias than those without no
dissatisfaction. Experiencing body dissatisfaction may increase empathy
towards individuals in larger bodies, thereby reducing their weight
bias. However, further examination is necessary to better understand
this relationship. Similarly, the two levels of education associated
with influencing the relationship between BMI and total AFAT scores
(some college and having a Master's Degree) are separated by two
additional levels of education (Associate Degree and Bachelor's Degree),
which makes drawing any conclusion about the relationship between
education, BMI, and anti-fat bias difficult. Thus, further research
examining this relationship is warranted as well.

Regardless of age, gender, body dissatisfaction, race, role in industry,
income, or education, total AFAT scores and all AFAT subscale scores
were below the anti-fat bias threshold 3. Dimmock et al.~(2009) reported
similar results with mean explicit weight bias values of 1.65, 2.66, and
2.92 for social, physical, and blame AFAT subscales. However,
conflicting research demonstrates that fitness professionals possess
strong implicit anti-fat biases (Dimmock et al., 2009; Fontana et al.,
2018; Robertson \& Vohora, 2008). The differences in findings could
result from measuring implicit versus explicit weight bias, where
implicit bias measures biases that emerge subconsciously without
awareness, and explicit bias measures conscious biases (Gawronski \&
Bodenhausen, 2006). A limitation of this study was the use of explicit
rather than implicit weight bias measures, which increases the
likelihood of response bias. The participants who chose to complete this
study have more favorable attitudes towards individuals in larger
bodies, which is why they were interested in participating. Future
research should examine how social identities and industry roles
influence implicit weight biases in the fitness industry.

Despite AFAT scores not being below the anti-fat bias threshold,
participants still reported anti-fat beliefs, particularly for the
physical and blame subscales. The blame AFAT score was consistently the
greatest of all of the AFAT subscales for each of the IVs, and
participants in the healthy BMI category had significantly greater AFAT
blame scores compared to individuals in the obese category. While this
is the first study to demonstrate such findings in fitness
professionals, Chambliss et al.~(2004) and Dimmock et al.~(2009)
reported similar, high blame subscale scores in exercise science
students and fitness center employees, respectively. In most fitness
certifications and curricula, there is a strong focus on preventing
obesity, where there is often an oversimplification of obesity, which
blames and stigmatizes individuals in larger bodies (Roehling et al.,
2007; Tesh \& Tesh, 1988). Interventions designed to reduce weight bias
by including information about the complex nature of obesity (e.g.,
uncontrollable causes of obesity) have been shown to successfully reduce
blame (Rukavina et al., 2010; Wijayatunga et al., 2019) and social
weight bias (Rukavina et al., 2010) in undergraduate students. While
future research examining the effect of similar interventions on weight
biases in fitness professionals might be useful, Gibson (2021) argues
that even fat activists who try to resolve individuals in larger bodies
of responsibility associated with their size inadvertently support the
notion of blame. By arguing that individuals in larger bodies who
exercise and eat well are naturally larger and ``innocent'' of their
body size (a term Gibson deems ``good fatty''), fat activists highlight
the notion that those in larger bodies who are not active or eating well
(``bad fatty) are''guilty'' and therefore to blame for their bodies.
Thus, the cognitive response of blame in relation to one's body size is
further exasperated.

An additional novel finding in this study was the difference in anti-fat
bias based on one's race. In opposition to Perez-Lopez et al.~(2001),
who reported greater anti-fat attitudes in White individuals, White
participants in this study reported significantly lower total anti-fat
bias compared to Black, Asian, and Hispanic participants. Similarly,
White participants reported significantly lower AFAT physical scores
compared to every other race. This finding contradicts Puhl et
al.~(2015), who found that in a U.S. sample, Black participants scored
lower on fat phobia and fat bias measures compared to White
participants. In the current study, breaking down weight bias
specifically into the physical subscale provides an important context
for the contradictory finding. The physical subscale represents how
attractive or unattractive one finds fat people. This is particularly
relevant for the sample population, where social norms within the
fitness industry uniquely prime individuals to have stronger weight
biases by framing weight loss goals to improve attractiveness.

Societally, we are primed to view thin, white bodies as more attractive,
which is rooted in a racist history (Strings, 2020). Research
demonstrates that greater exposure to anti-fat culture leads to stronger
anti-fat attitudes (Durso \& Latner, 2008; O'Brien et al., 2007), which
leads to internalizing such messages (Mensinger et al., 2016; Pearl et
al., 2019) However, more recent messaging around body positivity, which
aims to counter the views that only thin bodies are attractive and
worthy, has been criticized as a white feminist perspective (Johansson,
2021). If bodies other than the white, thin ideal are considered to be
unattractive, it is too much to challenge dominant stereotypes to accept
all other deviant aspects of non-white fat bodies (e.g., race, hair,
etc.); therefore, white women are the only group allowed the privilege
of viewing their own larger bodies as attractive or adopt self-love
(Johansson, 2021; Strings et al., 2019). This can then reinforce
internalized views of unattractiveness for non-white fitness
professionals, as they have not been able to receive the benefits of
body positivity under white supremacy.

In addition, other aspects of the fitness industry function under white
privilege. White fitness professionals are more likely to have control
and ownership over fitness spaces, which can make it challenging to
create spaces that counter racist attitudes (Strings, 2020; Strings et
al., 2019). People of color and those in larger bodies continue to be
underrepresented in the fitness industry and have likely experienced
many other biases themselves, which leads to a greater potential for
internalizing other biases prevalent in the industry.

Another area that needs continued exploration is the influence of one's
role in the industry on weight biases. While this paper sought to
understand this relationship, the small sample size within each role
made it difficult to make comparisons, which limited the analysis to
only comparing three roles. Additionally, most fitness professionals
have many roles in the industry (e.g., personal trainer and group
fitness instructor), which makes interpreting data difficult. It is also
possible that the type of certification (rather than a role in the
industry) influenced weight bias more.

\subsection{Conclusion}\label{conclusion}

This study reveals complex relationships in anti-fat biases among
fitness professionals. The interaction between BMI, gender, and
education level reveals intriguing patterns, challenging conventional
assumptions about weight biases within the fitness industry. Both
genders in the healthy BMI category express greater weight biases,
indicating societal favoritism towards smaller bodies. Despite
participants scoring below the defined anti-fat bias threshold, the
persistence of anti-fat beliefs, especially in the blame subscale, calls
attention to underlying biases not fully captured by explicit measures.
The blame directed towards individuals in larger bodies may be
perpetuated by oversimplified narratives surrounding obesity prevalent
in fitness certifications and curricula. The study also breaks new
ground by exploring racial disparities in anti-fat biases within the
fitness industry. The unexpected finding that White participants report
lower anti-fat biases challenges previous research, pointing to the
unique influence of the fitness industry's culture on shaping
perceptions. This underscores the importance of considering
industry-specific contexts in understanding weight biases.

In conclusion, this study unravels intricate dynamics of weight biases
in fitness professionals, challenging assumptions and emphasizing
industry-specific influences. It calls for ongoing research to
comprehensively understand the multifaceted factors shaping biases in
this unique professional domain. This study uses a diverse sample to
advance the current literature on weight bias concerns in the fitness
industry. This novel research is a necessary first step to future
research (e.g., intervention studies) that explores how biases in the
fitness industry influence the health behaviors of those seeking fitness
guidance.

\section{References}\label{references}

\phantomsection\label{refs}
\begin{CSLReferences}{1}{0}
\bibitem[\citeproctext]{ref-chambliss2004a}
Chambliss, H. O., Finley, C. E., \& Blair, S. N. (2004). Attitudes
toward obese individuals among exercise science students. \emph{Medicine
\& Science in Sports \& Exercise}, \emph{36}(3), 468--474.
\url{https://doi.org/10.1249/01.mss.0000117115.94062.e4}

\bibitem[\citeproctext]{ref-debarr2016a}
DeBarr, K., \& Pettit, M. (2016). Weight matters: Health educators'
knowledge of obesity and attitudes toward people who are obese.
\emph{American Journal of Health Education}, \emph{47}(6), 365--372.
\url{https://doi.org/10.1080/19325037.2016.1219282}

\bibitem[\citeproctext]{ref-dimmock2009a}
Dimmock, J. A., Hallett, B. E., \& Grove, R. J. (2009). Attitudes toward
overweight individuals among fitness center employees: An examination of
contextual effects. \emph{Research Quarterly for Exercise and Sport},
\emph{80}(3), 641--647.
\url{https://doi.org/10.1080/02701367.2009.10599603}

\bibitem[\citeproctext]{ref-durso2008a}
Durso, L. E., \& Latner, J. D. (2008). Understanding self‐directed
stigma: Development of the weight bias internalization scale.
\emph{Obesity}, \emph{16}(S2), 80--86.
\url{https://doi.org/10.1038/oby.2008.448}

\bibitem[\citeproctext]{ref-elran-barak2018a}
Elran-Barak, R., \& Bar-Anan, Y. (2018). Implicit and explicit anti-fat
bias: The role of weight-related attitudes and beliefs. \emph{Social
Science \& Medicine}, \emph{204}, 117--124.
\url{https://doi.org/10.1016/j.socscimed.2018.03.018}

\bibitem[\citeproctext]{ref-fontana2018a}
Fontana, F., Bopes, J., Bendixen, S., Speed, T., George, M., \& MACK, M.
(2018). Discrimination against obese exercise clients: An experimental
study of personal trainers. \emph{International Journal of Exercise
Science}, \emph{11}(5), 116.

\bibitem[\citeproctext]{ref-fontana2013a}
Fontana, F., Furtado Jr, O., Marston, R., Mazzardo Jr, O., \& Gallagher,
D. (2013). Anti-fat bias among physical education teachers and majors.
\emph{Physical Educator}, \emph{70}(1), 15.

\bibitem[\citeproctext]{ref-fontana2017a}
Fontana, F., Furtado Jr, O., Mazzardo Jr, O., Hong, D., \& Campos, W.
de. (2017). Anti-fat bias by professors teaching physical education
majors. \emph{European Physical Education Review}, \emph{23}(1),
127--138. \url{https://doi.org/10.1177/1356336X16643304}

\bibitem[\citeproctext]{ref-foster2003a}
Foster, G. D., Wadden, T. A., Makris, A. P., Davidson, D., Sanderson, R.
S., Allison, D. B., \& Kessler, A. (2003). Primary care physicians'
attitudes about obesity and its treatment. \emph{Obesity Research},
\emph{11}(10), 1168--1177. \url{https://doi.org/10.1038/oby.2003.161}

\bibitem[\citeproctext]{ref-fritz2012a}
Fritz, C. O., Morris, P. E., \& Richler, J. J. (2012). Effect size
estimates: Current use, calculations, and interpretation. \emph{Journal
of Experimental Psychology: General}, \emph{141}(1), 2.
\url{https://doi.org/10.1037/a0024338}

\bibitem[\citeproctext]{ref-gardner2010a}
Gardner, R. M., \& Brown, D. L. (2010). Body image assessment: A review
of figural drawing scales. \emph{Personality and Individual
Differences}, \emph{48}(2), 107--111.
\url{https://doi.org/10.1016/j.paid.2009.08.017}

\bibitem[\citeproctext]{ref-gawronski2006a}
Gawronski, B., \& Bodenhausen, G. V. (2006). Associative and
propositional processes in evaluation: An integrative review of implicit
and explicit attitude change. \emph{Psychological Bulletin},
\emph{132}(5), 692. \url{https://doi.org/10.1037/0033-2909.132.5.692}

\bibitem[\citeproctext]{ref-gibson2022a}
Gibson, G. (2022). Health (ism) at every size: The duties of the {``good
fatty.''} \emph{Fat Studies}, \emph{11}(1), 22--35.
\url{https://doi.org/10.1080/21604851.2021.1906526}

\bibitem[\citeproctext]{ref-gordon2020a}
Gordon, A. (2020). \emph{What we don't talk about when we talk about
fat}. Beacon Press.

\bibitem[\citeproctext]{ref-guidi2021a}
Guidi, J., Lucente, M., Sonino, N., \& Fava, G. A. (2021). Allostatic
load and its impact on health: A systematic review. \emph{Psychotherapy
and Psychosomatics}, \emph{90}(1), 11--27.
\url{https://doi.org/10.1159/000510696}

\bibitem[\citeproctext]{ref-hebl2001a}
Hebl, M. R., \& Xu, J. (2001). Weighing the care: Physicians' reactions
to the size of a patient. \emph{International Journal of Obesity},
\emph{25}(8), 1246--1252. \url{https://doi.org/10.1038/sj.ijo.0801681}

\bibitem[\citeproctext]{ref-heise2019a}
Heise, L., Greene, M. E., Opper, N., Stavropoulou, M., Harper, C.,
Nascimento, M., Zewdie, D., Darmstadt, G. L., Greene, M. E., \& Hawkes,
S. (2019). Gender inequality and restrictive gender norms: Framing the
challenges to health. \emph{The Lancet}, \emph{393}(10189), 2440--2454.
\url{https://doi.org/10.1016/S0140-6736(19)30652-X}

\bibitem[\citeproctext]{ref-johansson2021a}
Johansson, A. (2021). Fat, black and unapologetic: Body positive
activism beyond white, neoliberal rights discourses. In
\emph{Pluralistic struggles in gender, sexuality and coloniality:
Challenging swedish exceptionalism} (pp. 113--146).

\bibitem[\citeproctext]{ref-langdon2016a}
Langdon, J., Rukavina, P., \& Greenleaf, C. (2016). Predictors of
obesity bias among exercise science students. \emph{Advances in
Physiology Education}, \emph{40}(2), 157--164.
\url{https://doi.org/10.1152/advan.00185.2015}

\bibitem[\citeproctext]{ref-lewis1997a}
Lewis, R. J., Cash, T. F., \& Bubb‐Lewis, C. (1997). Prejudice toward
fat people: The development and validation of the antifat attitudes
test. \emph{Obesity Research}, \emph{5}(4), 297--307.
\url{https://doi.org/10.1002/j.1550-8528.1997.tb00555.x}

\bibitem[\citeproctext]{ref-marini2013a}
Marini, M., Sriram, N., Schnabel, K., Maliszewski, N., Devos, T.,
Ekehammar, B., Wiers, R., HuaJian, C., Somogyi, M., \& Shiomura, K.
(2013). Overweight people have low levels of implicit weight bias, but
overweight nations have high levels of implicit weight bias. \emph{PloS
One}, \emph{8}(12), 83543.
\url{https://doi.org/10.1371/journal.pone.0083543}

\bibitem[\citeproctext]{ref-maskedreference}
Masked Citation. (n.d.). \emph{Masked Title}.

\bibitem[\citeproctext]{ref-mensinger2016a}
Mensinger, J. L., Calogero, R. M., \& Tylka, T. L. (2016). Internalized
weight stigma moderates eating behavior outcomes in women with high BMI
participating in a healthy living program. \emph{Appetite}, \emph{102},
32--43. \url{https://doi.org/10.1016/j.appet.2016.01.033}

\bibitem[\citeproctext]{ref-milburn2019a}
Milburn, N. G., Beatty, L., \& Lopez, S. A. (2019). Understanding,
unpacking, and eliminating health disparities: A prescription for health
equity promotion through behavioral and psychological research---an
introduction. \emph{Cultural Diversity and Ethnic Minority Psychology},
\emph{25}(1), 1. \url{https://doi.org/10.1037/cdp0000266}

\bibitem[\citeproctext]{ref-obrien2007a}
O'Brien, K. S., Hunter, J. A., \& Banks, M. (2007). Implicit anti-fat
bias in physical educators: Physical attributes, ideology and
socialization. \emph{International Journal of Obesity}, \emph{31}(2),
308--314. \url{https://doi.org/10.1038/sj.ijo.0803398}

\bibitem[\citeproctext]{ref-pearl2019a}
Pearl, R., Himmelstein, M., Puhl, R., Wadden, T., Wojtanowski, A., \&
Foster, G. (2019). Weight bias internalization in a commercial weight
management sample: Prevalence and correlates. \emph{Obesity Science \&
Practice}, \emph{5}(4), 342--353. \url{https://doi.org/10.1002/osp4.354}

\bibitem[\citeproctext]{ref-perez2001a}
Perez‐Lopez, M. S., Lewis, R. J., \& Cash, T. F. (2001). The
relationship of antifat attitudes to other prejudicial and
gender‐related attitudes. \emph{Journal of Applied Social Psychology},
\emph{31}(4), 683--697.
\url{https://doi.org/10.1111/j.1559-1816.2001.tb01408.x}

\bibitem[\citeproctext]{ref-puhl2015a}
Puhl, R. M., Latner, J., O'Brien, K., Luedicke, J., Daníelsdóttir, S.,
\& Forhan, M. (2015). A multinational examination of weight bias:
Predictors of anti-fat attitudes across four countries.
\emph{International Journal of Obesity}, \emph{39}(7), 1166--1173.
\url{https://doi.org/10.1038/ijo.2015.32}

\bibitem[\citeproctext]{ref-robertson2008a}
Robertson, N., \& Vohora, R. (2008). Fitness vs. Fatness: Implicit bias
towards obesity among fitness professionals and regular exercisers.
\emph{Psychology of Sport and Exercise}, \emph{9}(4), 547--557.
\url{https://doi.org/10.1016/j.psychsport.2007.06.002}

\bibitem[\citeproctext]{ref-roehling2007a}
Roehling, M. V., Roehling, P. V., \& Pichler, S. (2007). The
relationship between body weight and perceived weight-related employment
discrimination: The role of sex and race. \emph{Journal of Vocational
Behavior}, \emph{71}(2), 300--318.
\url{https://doi.org/10.1016/j.jvb.2007.04.008}

\bibitem[\citeproctext]{ref-rukavina2010a}
Rukavina, P. B., Li, W., Shen, B., \& Sun, H. (2010). A service learning
based project to change implicit and explicit bias toward obese
individuals in kinesiology pre-professionals. \emph{Obesity Facts},
\emph{3}(2), 117--126. \url{https://doi.org/10.1159/000302794}

\bibitem[\citeproctext]{ref-sartore2007a}
Sartore, M. L., \& Cunningham, G. B. (2007). Weight discrimination,
hiring recommendations, person--job fit, and attributions:
Fitness-industry implications. \emph{Journal of Sport Management},
\emph{21}(2), 172--193. \url{https://doi.org/10.1123/jsm.21.2.172}

\bibitem[\citeproctext]{ref-schwartz2003a}
Schwartz, M. B., Chambliss, H. O., Brownell, K. D., Blair, S. N., \&
Billington, C. (2003). Weight bias among health professionals
specializing in obesity. \emph{Obesity Research}, \emph{11}(9),
1033--1039. \url{https://doi.org/10.1038/oby.2003.142}

\bibitem[\citeproctext]{ref-strings2020a}
Strings, S. (2020). \emph{Fearing the black body: The racial origins of
fat phobia}. Oxford University Press.
\url{https://doi.org/10.1177/0094306120915912nn}

\bibitem[\citeproctext]{ref-strings2019a}
Strings, S., Headen, I., \& Spencer, B. (2019). Yoga as a technology of
femininity: Disciplining white women, disappearing people of color in
yoga journal. \emph{Fat Studies}, \emph{8}(3), 334--348.
\url{https://doi.org/10.1080/21604851.2019.1583527}

\bibitem[\citeproctext]{ref-sutin2013a}
Sutin, A. R., \& Terracciano, A. (2013). Perceived weight discrimination
and obesity. \emph{PloS One}, \emph{8}(7), 70048.
\url{https://doi.org/10.1371/journal.pone.0070048}

\bibitem[\citeproctext]{ref-tesh1988a}
Tesh, S. N., \& Tesh, S. (1988). \emph{Hidden arguments: Political
ideology and disease prevention policy}. Rutgers University Press.

\bibitem[\citeproctext]{ref-vartanian2011a}
Vartanian, L. R., \& Novak, S. A. (2011). Internalized societal
attitudes moderate the impact of weight stigma on avoidance of exercise.
\emph{Obesity}, \emph{19}(4), 757--762.
\url{https://doi.org/10.1038/oby.2010.234}

\bibitem[\citeproctext]{ref-washington2011a}
Washington, R. L. (2011). Childhood obesity: Issues of weight bias.
\emph{Preventing Chronic Disease}, \emph{8}(5).

\bibitem[\citeproctext]{ref-wijayatunga2019a}
Wijayatunga, N. N., Kim, Y., Butsch, W. S., \& Dhurandhar, E. J. (2019).
The effects of a teaching intervention on weight bias among kinesiology
undergraduate students. \emph{International Journal of Obesity},
\emph{43}(11), 2273--2281.
\url{https://doi.org/10.1038/s41366-019-0325-0}

\bibitem[\citeproctext]{ref-zaroubi2021a}
Zaroubi, L., Samaan, T., \& Alberga, A. S. (2021). Predictors of weight
bias in exercise science students and fitness professionals: A scoping
review. \emph{Journal of Obesity}.
\url{https://doi.org/10.1155/2021/5597452}

\end{CSLReferences}





\end{document}
